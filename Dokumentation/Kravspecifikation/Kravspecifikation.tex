

\chapter{Kravspecifikation}




\subsection{ Kravspecifikation (sigtes i mod)}
Der lægges op til, at der udvikles to vandmålerprototyper - en baseret på den gængse teknologi, og en baseret på ny teknologi udviklet af Miitors. Da information om den nye teknologi endnu ikke er blevet gjort til rådighed, fokuseres der i første omgang på udvikling af den første prototype.



\subsection{Funktionelle krav}
\begin{itemize}
\item[Flowmåler skal kunne måle vand med en ledningsevne over 800uS/cm. \citep{rentvand}]	
\end{itemize}


\subsection{Ikke-funktionelle krav}
\begin{itemize}
\item[Skal måle flow baseret på magnetisk-induktiv målerteknologi.]
\item[Skal overholde de metrologiske krav beskrevet i OIML R 49-1:2013 sektion 4.]
\item[Skal have Q3 (permanent flow rate) på 4 m3/t.]
\item[Skal have et dynamikområde på R = Q3/Q1 = 800.]
\item[Skal have DN20 som rørtykkelse.]
\item[Bør have et display.]
\item[Lav energi forbrug]
\item[Skal kunne transmitter over uart]
\end{itemize}





\subsection{Underlæggen krav}


OIML RO49 Krav
\label{OIML RO49 Krav}

\begin{table}[H]
        \centering
        \begin{tabularx}{\linewidth}{|p{120pt}|X |X|}
                      \hline
        \textbf{OIML RO49 References}   &\textbf{OIML R49 Water Custody Transfer Certificate}.   &
        \\\hline
        
        &     &
        \\\hline
        
 
        
\end{tabularx}
\caption{Ordliste}
\label{tab:Ordliste}
\end{table}






System Demands 
\label{System Demands }

\begin{table}[H]
        \centering
        \begin{tabularx}{\linewidth}{|p{20pt}|X|X |X|}
                      \hline
                      
   &\textbf{Description} &\textbf{Demanded} &\textbf{Coment} 
        \\\hline
        
        &	&  &	  \\ 
  Q1  &	Minimum flow rate.&	&	   
        
		\\\hline

            
  Q2  &	Transitional flow rate.&	&	   
        
		\\\hline
		        
   Q3 &Permanent flow rate.	&	$$ 4 m^{3}/h$$&	   
        
		\\\hline
		        
  Q4  &Overload flow rate.	&	&	   
        
		\\\hline
		        
 	  	 &	&  &	  \\ 
   DT &Tube inside diameter.  	& DN20=20mm=$\dfrac{3}{4}$’’&	   
        
		\\\hline
		 &	&  &	  \\ 
    &	Battery lifetime. &TBD	& With the typical sample rate. 	   
        
		\\\hline
		 &	&  &	  \\   
    &Measuring/sample rate typical. 	&1/4 sample/sekunt.	&	   
        
		\\\hline    
      
        
    &Measuring/sample rate maximum. 	&	>10 sample/sekunt.&	   
        
		\\\hline    
      
                      
    &	Measuring/sample rate minimum.&< 1/60 sample/sekunt.	&	   
        
		\\\hline    
      
                
    &	&	&	   
        
		\\\hline    
      

\end{tabularx}
\caption{System Demands }
\label{tab:System_Demands }
\end{table}

















SubSystem Demand 
\label{SubSystem Demand}

\begin{table}[H]
        \centering
        \begin{tabularx}{\linewidth}{|p{20pt}|X|X |X|}
                      \hline
                      
   &\textbf{Description} &\textbf{Demanded} &\textbf{Coment} 
        \\\hline
   		 &	&  &	  \\ 
       & \textbf{MicroController.}  & &
        
        \\\hline    
  
    
    &ADC.	&14-16 bit fast	&	   
        
		\\\hline    
      
                   &	&	&	   
        
		\\\hline     
                  

   		 &	&  &	  \\ 
       &\textbf{Interfaces.}  & &
        
        \\\hline          
                  
                      
    &	&	&	   
        
		\\\hline    
		                
    &	&	&	   
        
		\\\hline  
		
   		 &	&  &	  \\ 
       &\textbf{Input Amplifier.}  & &
        
        \\\hline     		  
		                
    &BandWidth.	&TBD	&Probably In the kHz range because of the alternating magnetic field.	   
        
		\\\hline    
		                
    &Dynamic input voltage range. 	&TBD	&Voltage expected from the electrodes.	   
        
		\\\hline    
		                
    &Dynamic output voltage range.	&TBD	&Dependent on the ADC.	   
        
		\\\hline    
		                
    &Input impedance.	&TBD/TBM	&Dependent on the liquid flowing through the system. 	   
        
		\\\hline    
		                
    &	&	&	   
        
		\\\hline    

\end{tabularx}
\caption{SubSystem Demand}
\label{tab:SubSystem Demand }
\end{table}













